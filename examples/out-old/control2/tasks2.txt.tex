\documentclass[11pt,a4paper]{article}

\usepackage{fullpage}
\usepackage[utf8]{inputenc}
\usepackage[russian]{babel}
\usepackage{graphicx}
\usepackage{float}
\usepackage[stable]{footmisc}
\usepackage{caption}
\usepackage{subcaption}
\usepackage{url}
\usepackage{amsmath}
\usepackage{amssymb}
\usepackage{listings}

\usepackage[margin=0.4in]{geometry}

\setlength\parindent{0pt}

\pagenumbering{gobble}

\begin{document}

%
% Вариант 1
%

\begin{center}
  \textbf{\LARGE{Вариант 1 (КР 2)} \\}

  Арсений Серока
\end{center}

Для каждого задания требуется также придумать несколько разумных тестов. Хорошие тесты могут улучшить оценку задания. Также требуется перед каждой задачей в комментарии писать текст задания (можно коротко, главное, чтобы было понятно, какое задание). Ваша контрольная работа должна быть оформлена как stack проект. Тесты пишите в соответствующем Main модуле. Их должно быть возможно запустить при помощи команды \texttt{stack exec}. Требуется соблюдать все замечания к текущим домашним заданиям.

\begin{figure}[H]
  \centering
  \includegraphics[width=500pt]{images/bord3.png}
\end{figure}

\begin{enumerate}
  \item[1.]
  \textbf{\textit{Условие:}}

Реализуйте требуемую функциональность из задания. Запрещается использовать \texttt{do}-синтаксис.

  \textbf{\textit{Задача:}}

Задание 1
  \item[2.]
  \textbf{\textit{Условие:}}

Решите задачу, используя трансформеры монад. Можно не использовать трансформеры, но тогда балл будет сильно ниже. Постарайтесь разделить чистый код и код с сайд-эффектами. Разрешается использовать \texttt{do}-синтаксис. Реализация также должна быть как можно более короткой и эффективной.

  \textbf{\textit{Задача:}}

Задание 3
\end{enumerate}

\begin{figure}[H]
  \centering
  \includegraphics[width=500pt]{images/bord4.png}
\end{figure}

\pagebreak

%
% Вариант 2
%

\begin{center}
  \textbf{\LARGE{Вариант 2 (КР 2)} \\}

  Дмитрий Коваников
\end{center}

Для каждого задания требуется также придумать несколько разумных тестов. Хорошие тесты могут улучшить оценку задания. Также требуется перед каждой задачей в комментарии писать текст задания (можно коротко, главное, чтобы было понятно, какое задание). Ваша контрольная работа должна быть оформлена как stack проект. Тесты пишите в соответствующем Main модуле. Их должно быть возможно запустить при помощи команды \texttt{stack exec}. Требуется соблюдать все замечания к текущим домашним заданиям.

\begin{figure}[H]
  \centering
  \includegraphics[width=500pt]{images/bord3.png}
\end{figure}

\begin{enumerate}
  \item[1.]
  \textbf{\textit{Условие:}}

Реализуйте требуемую функциональность из задания. Запрещается использовать \texttt{do}-синтаксис.

  \textbf{\textit{Задача:}}

Задание 2
  \item[2.]
  \textbf{\textit{Условие:}}

Решите задачу, используя трансформеры монад. Можно не использовать трансформеры, но тогда балл будет сильно ниже. Постарайтесь разделить чистый код и код с сайд-эффектами. Разрешается использовать \texttt{do}-синтаксис. Реализация также должна быть как можно более короткой и эффективной.

  \textbf{\textit{Задача:}}

Задание 3
\end{enumerate}

\begin{figure}[H]
  \centering
  \includegraphics[width=500pt]{images/bord4.png}
\end{figure}

\pagebreak

\end{document}