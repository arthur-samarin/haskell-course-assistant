\documentclass[11pt,a4paper]{article}

\usepackage{fullpage}
\usepackage[utf8]{inputenc}
\usepackage[russian]{babel}
\usepackage{graphicx}
\usepackage{float}
\usepackage[stable]{footmisc}
\usepackage{caption}
\usepackage{subcaption}
\usepackage{url}
\usepackage{amsmath}
\usepackage{amssymb}
\usepackage{listings}

\usepackage[margin=0.4in]{geometry}

\setlength\parindent{0pt}

\pagenumbering{gobble}

\begin{document}

%
% Вариант 1
%

\begin{center}
  \textbf{\LARGE{Вариант 1 (КР 1)} \\}

  Арсений Серока
\end{center}

Для каждого задания требуется также придумать несколько разумных тестов. Хорошие тесты могут улучшить оценку задания. Также требуется перед каждой задачей в комментарии писать текст задания (можно коротко, главное, чтобы было понятно, какое задание). Ваша контрольная работа должна быть оформлена как stack проект. Тесты пишите в соответствующем Main модуле. Их должно быть возможно запустить при помощи команды \texttt{stack exec}. Требуется соблюдать все замечания к текущим домашним заданиям.

\begin{figure}[H]
  \centering
  \includegraphics[width=500pt]{images/bord3.png}
\end{figure}

\begin{enumerate}
  \item[1.]
  \textbf{\textit{Условие:}}

Написать функцию в две строки: тип в наиболее общем виде и реализация. Реализация должна быть максимально короткой. Нельзя использвать другие пакеты кроме \texttt{Prelude}, \texttt{Data.List}, \texttt{Data.Char}, \texttt{Data.Map}, \texttt{Data.Tree}.

  \textbf{\textit{Задача:}}

Задание 2
  \item[2.]
  \textbf{\textit{Условие:}}

Как в прошлом задании, только реализация может занимать произвольное число строк, хотя для максимального балла требуется наиболее короткая (форматирование и длина имён переменных не учитывается). Решение должно быть асимптотически эффективным, причём скрытую константу в асимптотике тоже желательно уменьшить, насколько это возможно.

  \textbf{\textit{Задача:}}

Задание 3
  \item[3.]
  \textbf{\textit{Условие:}}

Реализовать структуру данных.

  \textbf{\textit{Задача:}}

Сложное задание 3
  \item[4.]
  \textbf{\textit{Условие:}}

Придумать и реализовать заданный тайпкласс.

  \textbf{\textit{Задача:}}

Очень сложное задание 1
\end{enumerate}

\begin{figure}[H]
  \centering
  \includegraphics[width=500pt]{images/bord4.png}
\end{figure}

\pagebreak

%
% Вариант 2
%

\begin{center}
  \textbf{\LARGE{Вариант 2 (КР 1)} \\}

  Дмитрий Коваников
\end{center}

Для каждого задания требуется также придумать несколько разумных тестов. Хорошие тесты могут улучшить оценку задания. Также требуется перед каждой задачей в комментарии писать текст задания (можно коротко, главное, чтобы было понятно, какое задание). Ваша контрольная работа должна быть оформлена как stack проект. Тесты пишите в соответствующем Main модуле. Их должно быть возможно запустить при помощи команды \texttt{stack exec}. Требуется соблюдать все замечания к текущим домашним заданиям.

\begin{figure}[H]
  \centering
  \includegraphics[width=500pt]{images/bord3.png}
\end{figure}

\begin{enumerate}
  \item[1.]
  \textbf{\textit{Условие:}}

Написать функцию в две строки: тип в наиболее общем виде и реализация. Реализация должна быть максимально короткой. Нельзя использвать другие пакеты кроме \texttt{Prelude}, \texttt{Data.List}, \texttt{Data.Char}, \texttt{Data.Map}, \texttt{Data.Tree}.

  \textbf{\textit{Задача:}}

Задание 1
  \item[2.]
  \textbf{\textit{Условие:}}

Как в прошлом задании, только реализация может занимать произвольное число строк, хотя для максимального балла требуется наиболее короткая (форматирование и длина имён переменных не учитывается). Решение должно быть асимптотически эффективным, причём скрытую константу в асимптотике тоже желательно уменьшить, насколько это возможно.

  \textbf{\textit{Задача:}}

Задание 3
  \item[3.]
  \textbf{\textit{Условие:}}

Реализовать структуру данных.

  \textbf{\textit{Задача:}}

Сложное задание 1
  \item[4.]
  \textbf{\textit{Условие:}}

Придумать и реализовать заданный тайпкласс.

  \textbf{\textit{Задача:}}

Очень сложное задание 2
\end{enumerate}

\begin{figure}[H]
  \centering
  \includegraphics[width=500pt]{images/bord4.png}
\end{figure}

\pagebreak

\end{document}